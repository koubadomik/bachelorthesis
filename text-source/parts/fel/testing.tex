\chapter{Testování} \label{chap:testing}
V následující kapitole rozebereme testování.
% Klientská část slouží pouze pro demonstraci funkčnosti serverové, nejedná se o plnohodnotnou aplikaci, není třeba jí testovat.
Kapitola bude mít dvě části. V první z nich rozebereme obecně, jakým způsobem lze testovat hexagonální architekturu. V druhé části demonstrujeme některé z představených druhů testů na našem prototypu.
\section{Testování hexagonální architektury}
Alistar Cockburn, který hexagonální architekturu nejvíce prosadil, zmiňuje jako jeden z hlavních přínosů této architektury právě izolované testování. Primárně mluví o automatizovaném testování domény, odstíněném od ostatní infrastruktury (databáze, API jiných systémů) \cite{cockburn}. Doména obsahuje hlavní případy užití, je velmi výhodné je testovat izolovaně od všeho ostatního a hlavně automatizovaně.\par
\noindent Testování hexagonální architektury může probíhat na všech částech systému, nejen na doméně.
\paragraph{Rozdělení testování:}
\begin{itemize}
    \item Testování domény
    \item Testování SPI
    \item Testování API
\end{itemize}
\subsection{Testování domény}
Testy domény jsou architektonicky na stejné úrovni jako jsou API adaptéry, ve své podstatě se jedná o API adaptéry \cite{cockburn}. Tyto adaptéry volají implementaci API portů, které obsahují logiku, a tuto logiku testují. Logika (případy užití) je dále nabízena ostatním systémům, je tedy nutné jí otestovat.\par
V případě tohoto druhu testování je důležité doménu izolovat od ostatních vlivů, v podstatě se jedná o jednotkové testování. Ostatními vlivy jsou v tomto kontextu myšleny SPI adaptéry, které můžeme nahradit například mock objekty \cite{cockburn}. Jinými slovy můžeme SPI adaptéry, které slouží integraci s jinými systémy, nahradit dalšími adaptéry, které budou tuto integraci pouze předstírat a budou vracet námi definovaný správný výsledek.
\subsection{Testování SPI}
Poté co zajistíme doménu, je třeba zajistit, že SPI adaptéry budou poskytovat všechna data ve správné podobě (v podobě jako v mock objektech).\par
SPI adaptéry resp. moduly, které obsahují SPI adaptéry, mohou mít libovolnou vnitřní architekturu. Tuto architekturu je třeba testovat tak, jak je pro ní typické, a tak zajistit funkčnost SPI adaptérů.\par
Na klasické vrstevnaté architektuře (obdoba brownovy architektury \ref{tab:brown}) lze například provádět testování po jednotlivých vrstvách jako je tomu zvykem. Obvyklé je využití mock objektů, in-memory databází a obdobných technik.
\subsection{Testování API}
Posledním modulem, který zbývá jsou API adaptéry. Existují různé druhy těchto adaptérů, pro testování bereme v úvahu hlavně REST a GraphQL.\par
Tento druh testování se v případě malé aplikace může provést ručně. V komplikovanějších případech existují různé testovací knihovny a frameworky (například HttpClient - \url{http://hc.apache.org/httpcomponents-client-ga/index.html}). Tyto knihovny typicky volají API systému a kontrolují funkčnost, jako například návratové kódy.

\section{Testování implementované aplikace}
Prototyp aplikace, který jsme v rámci této práce vyvinuli se nejvíce opírá o modul \ctulst(none)!gdb_solution!, na ten je třeba se zaměřit při testování nejintenzivněji.\par
\noindent Oblasti naší aplikace k testování:
\begin{itemize}
    \item Persistentní vrstva modulu \ctulst(none)!gdb_solution!
    \item Logika modulu \ctulst(none)!gdb_solution!
    \item REST API
\end{itemize}
Samozřejmě je možné zapojit další druhy testů, vzhledem k rozsahu prototypu jsou však tyto tři úrovně dostačující.
% Hlavním důvodem, proč jsme další testy neimplementovali je fakt, že další integrace mezi moduly jsou bez jakékoliv složitější logiky vhodné k testování.
\section{Průběh testu}
Testy jsou v naší aplikaci vždy rozděleny na tři části:
\begin{itemize}
    \item \textit{arrange} - příprava akce (inicializace instancí, jiné integrační akce a podobně)
    \item \textit{act} - testovaná akce
    \item \textit{assert} - ověření výsledku
\end{itemize}
\subsection{Testování persistentní vrstvy}
Integrační testy persistentní vrstvy jsou v kontextu naší práce klíčové. Hlavně proto, že nepoužíváme jiné datové úložiště než graphDB. Navíc hlavní dotaz pro vyhledávání osob dle znalosti probíhá pomocí SPARQL na této úrovni.\par
Při testování integrace se používá testovací repozitář GraphDB. Hlavním důvodem je, že integrační testy musí mít izolované prostředí, nezávislé na ostatních (produkčních a podobně).
\subsubsection{Pokrytí testy}
Jelikož je modul \ctulst(none)!gdb_solution/dao! naprogramován genericky, je poměrně přímočaré otestovat společné metody dao objektů (CRUD). Individuálně se testují jen specifické funkce jednotlivých dao objektů, v našem případě jen \ctulst(none)!PersonDao!.
\subsubsection{Příklad integračního testu s popisem}
Níže je k vidění příklad integračního testu, který testuje metodu \ctulst(none)!findAll!. Test probíhá nad entitou \ctulst(none)!PersonGDB!, nicméně se jedná o generickou metodu, testujeme tedy i ostatní entity. V části \textit{arrange} připravujeme prostředí, tedy vytváříme tři instance entity osoba a zapisujeme je do testovacího repozitáře graphDB. V části \textit{act} čteme všechny záznamy osob z repozitáře graphDB. V části \textit{assert} kontrolujeme obsah výsledku s obsahem původního seznamu, používáme k tomu funkce \ctulst(none)!assertEquals! a \ctulst(none)!assertTrue! z knihovny JUnit 4 (\url{https://junit.org/junit4/}). 
\begin{lstlisting}[language=JAVA, caption= Příklad integračního testu, captionpos=b]
    @Test
    public void findAllReturnsAllExistingInstances() {
        //arrange
        final List<PersonGDB> persons = new ArrayList<>();
        persons.add(new PersonGDB("Dominik", "Kouda", "koudadom"));
        persons.add(new PersonGDB("Dominik2", "Kouda1", "koudadom1"));
        persons.add(new PersonGDB("Dominik1", "Kouda2", "koudadom2"));
        //act
        personDao.persist(persons);
        final List<PersonGDB> result = personDao.findAll();
        //assert
        assertEquals(persons.size(), result.size());
        boolean found = false;
        for (PersonGDB person : persons) {
            for (PersonGDB person2 : result) {
                if (person.equals(person2)) {
                    found = true;
                    break;
                }
            }
            assertTrue(found);
        }
    }
\end{lstlisting}
% [TODO: možná doplnit test metody getPeopleByKnowledge]
\subsection{Testování logiky}
Logikou je v modulu \ctulst(none)!gdb_solution! myšlena hlavně třída \ctulst(none)!KnowledgeDTOList!. Tato třída reprezentuje množinu znalostí. Množina má navíc tu vlastnost, že v případě vložení duplicitního záznamu, zůstane v množině vždy prvek s větší sílou znalosti. Tato třída se využívá při vyhledávání osob dle znalosti ke kompletaci výsledku SPARQL dotazu.\par
V testu níže lze vidět obdobný postup jako tomu bylo výše. V tomto konkrétním příkladu kontrolujeme, že při přidání jedné osoby vícekrát se aktualizuje její síla znalosti, byla-li přidávána vyšší.
\begin{lstlisting}[language=JAVA, caption= Jednotkový test aplikační logiky modulu gdb\_solution, captionpos=b]
    @Test
    public void addsOnlyHighest() {
        //arrange
        KnowledgeDTOList list = new KnowledgeDTOList();
        PersonGDB p1 = new PersonGDB("Dominik", "Kouda", "koudadom");
        PersonGDB p2 = new PersonGDB("Dominik2", "Kouda1", "koudadom1");
        PersonGDB p3 = new PersonGDB("Dominik1", "Kouda2", "koudadom2");
        //act
        list.add(new KnowledgeDTO(p1, 0.5, new HashSet<>()));
        list.add(new KnowledgeDTO(p2, 0.5, new HashSet<>()));
        list.add(new KnowledgeDTO(p3, 0.6, new HashSet<>()));
        list.add(new KnowledgeDTO(p1, 0.7, new HashSet<>()));
        //assert
        assertEquals(3, list.size());
        boolean success = false;
        for(KnowledgeDTO dto: list.getKnowledgeDTOs()){
            if(dto.getOwner().equals(p1) && 
                    dto.getWeightOfKnowledge() == 0.7){
                success = true;
            }
        }
        assertTrue(success);
    }
}
\end{lstlisting}
Tímto druhem testů je otestována správná funkce této třídy. Konkrétně je to kontrola duplicit v množině a maximalizace síly znalosti každé osoby v množině, při postupném přidávání.
\subsection{Testování REST API}
API má dva základní případy použití - vrácení seznamu všech konceptů a vyhledání osob dle zvoleného konceptu. Vzhledem k tomu, že tato část aplikace není příliš složitá, rozhodli jsme se pro ruční otestování.\par
\subsubsection{Scénáře}
\begin{table}[h!]
\begin{ctucolortab}
% \begin{tabular}{|l|l|m{3cm}|l|m{2,75cm}|}
\begin{tabularx}{0.95\linewidth}{X|X|X}
\bfseries Testovaný případ  & \bfseries Očekávaný výsledek & \bfseries Výsledek testu\\\hline 
    % \hline
    %  & Forma dat & Dostupnost dat & Aktualizováno & Správce \\ \hline
    Dotaz na všechny koncepty & Seznam všech konceptů ve formátu JSON & OK \\\hline
    Vyhledání dle konceptu (existujícího) & Seznam osob ve formátu JSON & OK \\\hline
    Vyhledání dle konceptu (neexistujícího) & Status 404 & Nalezena chyba \\
\end{tabularx}
\end{ctucolortab}
	\caption{Tabulka testovacích případů pro REST API}
	\label{tab:rest-testing}
\end{table}
Testování proběhlo bez větších problémů, byla nalezena chyba při dotazu na neexistující koncept. Chyba byla opravena.
\section{Shrnutí}
V této kapitole jsme shrnuli testování naší aplikace. V úvodu jsme popsali výhody testování hexagonální architektury a používané principy. V druhé části kapitoly jsme popsali testování serverové části našeho prototypu. Vzhledem k rozsahu prototypu jsme nepoužili všechny druhy testů, které je možné pro hexagonální architekturu implementovat.\par
Implementace testů proběhla bez problémů, nalezené chyby jsme opravili. Ruční testování proběhlo též bez komplikací, nalezené chyby jsme opravili.





% \subsection{Další výhody testování}
% Může být například testovací profil jen pro účely UI testování, který nezdržuje žádná databáze apod. zkrátka \cite{cockburn}

% \begin{itemize}
%     \item SPI - infrastrukturu lze nahradit různými způsoby - dummy, mock (viz fowler), jiný zdroj (jiný typ databáze...), skutečná databáze s testovacími daty
%     \item API - Lze nahradit testovací frameworkem (testování domény, je v podstatě na stejné úrovni jako je samotné API, je v podstatě API adaptérem)
% \end{itemize}





% Jak jsme již napsali v předchozí kapitole \ref{chap:implementation}: Účelem implementační části této práce je dokázat, že navržené řešení je možno použít a že skutečně může po dalších rozšířeních fungovat v praxi. K testování toho není mnoho, proto jsme zvolili pouze testovací úrovně, které dle nás mají pro náš rozsah smysl.\par
% Důležité je zmínit, že zvolená hexagonální architektura patří mezi nejlépe testovatelné, obzvlášť díky své modularitě a zapouzdření.
% % testování hexagonální architektury teoreticky (jeden zdroj tam byl)



