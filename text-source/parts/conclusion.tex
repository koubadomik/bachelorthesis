\chapter{Závěr}
Tato práce měla tři základní cíle. Prvním cílem bylo \textit{zanalyzovat problematiku vyhledávání osob a pracovišť dle kompetencí}. Tento cíl jsme splnili v první části, kdy jsme celý problém rozdělili na dílčí úkoly, a ty jsme postupně rozebrali. Začali jsme konceptuálním modelem celého problému, ze kterého jsme byli schopni lépe pochopit problematiku. Dále jsme zanalyzovali možnosti reprezentace kompetencí resp. znalostí, postupovali jsme od těch jednodušších (sémantické sítě) k těm složitějším (deskripční logika). V této části jsme jako nejvhodnější reprezentaci pro náš případ zvolili ontologie.  Hlavně proto, že jako jediný formální jazyk pro reprezentaci znalostí mají praktickou podporu technologií jako jsou RDF, OWL a SKOS. \par
Dále jsme zanalyzovali možné datové zdroje, zejména zdroje znalostí osob a znalostní báze. V této části jsme zjistili, že znalosti osob lze v akademickém prostředí nejlépe získat prostřednictvím vědeckých publikací a jiných výstupů vědecké činnosti. V komerčním prostředí se pro naše účely mohou vyskytovat znalostní systémy. Nejrozsáhlejší zdroje znalostní báze jsou na globální úrovni DBpedia a WordNet. Zjistili jsme také, že zdroje strukturovaných dat jsou často velmi omezené a že práce s nestrukturovanými daty jako je prostý text vyžaduje další zpracování (sémantická anotace a podobně).\par
Nakonec jsme se zabývali zpracováním dat. Rozebrali jsme jejich transformaci do ontologií pomocí nástrojů jako je OntoRefine. Následně jsme rozebrali jazyk SPARQL, který je klíčový v oblasti práce s ontologickými daty (RDF triply) a umožňuje provádět CRUD operace. V závěru této sekce jsme ještě zmínili obecný způsob získání síly znalosti, která je klíčovou metrikou při porovnávání znalostí více osob.
\par
Druhý cíl \textit{navrhnout systém pro vyhledávání osob a pracovišť dle kompetencí a provést rešerši dostupných dat} jsme naplnili ve druhé části práce. Nejdříve jsme rozebrali datové zdroje dostupné na FEL ČVUT. Klíčové jsou v tomto ohledu systémy Usermap a V3S. Systém Usermap může díky svému API sloužit jako zdroj osob a pracovišť. Systém V3S obsahuje rozsáhlou databázi vědecké činnosti na ČVUT. Dalším zjištěním v této části práce bylo, že na ČVUT neexistuje žádné jednotné rozhraní pro získávání nevědeckých dat o zaměstnancích - životopisy, osobní stránky a podobně.\par
Na analýzu datových zdrojů jsme navázali návrhem systému pro vyhledávání dle kompetencí. Systém jsme rozebrali počínaje požadavky, hlavním z nich je funkční požadavek na samotné vyhledávání, další důležitý je implementační požadavek na modularitu celého systému. Důraz na modularitu klademe, protože vyhledávání dle kompetencí je komplikovaný problém a není zjevné, kolik možných řešení existuje a jak je lze kombinovat.\par
Dále jsme postupovali přes případy užití a wireframy k výběru architektury. Požadavek na modularitu systému byl hlavním důvodem, proč jsme se odklonili od klasické vrstevnaté architektury. Nejdříve jsme rozebrali cibulovou architekturu, která je vzhledem ke svému nízkému provázání (angl. decoupling) vhodným kandidátem pro systémy jako je ten náš. Nakonec jsme zvolili variantu této architektury, totiž architekturu hexagonální. Tato architektura navíc ještě definuje způsob komunikace pomocí portů a adaptérů, což se v našem případě hodí. Hexagonální architekturu jsme tedy zevrubně představili a následně jí aplikovali na náš případ.\par
V poslední části práce jsme rozebrali prototyp navržené aplikace, který jsme implementovali. Prototyp má dvě části, server a klient. Demonstrovali jsme na něm funkčnost navržené aplikace. Jako technologii pro serverovou část jsme zvolili programovací jazyk JAVA, který vyhovuje technologickému prostředí ČVUT FEL. Pro ukládání dat pomocí ontologií jsme využili existující \textit{triple-store} GraphDB, jako schéma jsme použili ontologii SKOS rozšířenou o několik našich entit. Nakonec jsme v poslední kapitole rozebrali testování hexagonální architektury, nejdříve obecně a následně jsme některé z technik demonstrovali na našem prototypu.\par
Splnili jsme všechny stanovené cíle. Rozebrali jsme zevrubně celý problém vyhledávání dle kompetencí, navrhli systém, který tuto problematiku řeší a implementovali a otestovali prototyp pro FEL ČVUT. Nakonec jsme v kapitole budoucí práce shrnuli, jak na tuto práci navázat. Největší otázkou pro budoucí práce zůstává transformace existujících dat do zvolených struktur, do RDF triplů, a jejich následné začlenění do systému.






% Tato práce měla tři základní cíle. Prvním cílem bylo \textit{zanalyzovat problematiku vyhledávání osob a pracovišť dle kompetencí}. Tento cíl jsme splnili v první části této práce, kdy jsme celý problém rozdělili na dílčí úkoly a ty jsme postupně zanalyzovali. Klíčovou byla v této části volba reprezentace kompetencí resp. znalostí. Rozhodli jsme se je reprezentovat pomocí ontologického formálního jazyka. Při řešení dalších dílčích úkolů v první části jsme provedli rozbor každého problému a rešerši existujících technologií.\par
% Druhý cíl \textit{navrhnout systém pro vyhledávání osob a pracovišť dle kompetencí a provést rešerši dostupných dat} jsme naplnili v druhé části práce, konkrétně potom v kapitolách \ref{chap:data-sources-analysis} a \ref{chap:app-design}. V těchto kapitolách jsme zevrubně popsali potenciální datové zdroje a popsali jsme návrh systému.\par
% Implementovaný prototyp je přiložen k této práci na CD nosiči (příloha \ref{app:cd-content}). Dokumentace a popis prototypu jsou obsaženy v kapitolách \ref{chap:implementation} a \ref{chap:testing}. Tyto výstupy náleží k poslednímu cíli této práce - \textit{implementovat prototyp systému pro vyhledávání osob a pracovišť dle kompetencí pro FEL ČVUT}.\par
% Způsob jakým na tuto práci navázat a rozvinout její výstupy (implementační i teoretické) jsme shrnuli v kapitole č. \ref{future}.

% [Propojit to nějak s pokyny k vypracování]


% Cílem naší práce bylo vyřešit problematiku vyhledávání osob a pracovišť dle kompetencí na FEL ČVUT.\par
% Celou problematiku jsme v první části práce rozebrali v jednotlivých dílčích krocích. Klíčovou byla v této části volba reprezentace kompetencí resp. znalostí. Rozhodli jsme se je reprezentovat pomocí ontologického formálního jazyka. Při řešení dalších dílčích úkolů v první části jsme provedli rozbor každého problému a rešerši existujících technologií.\par
% Na první část jsme navázali v druhé polovině práce. Zde jsme již přistoupili ke konkrétnímu návrhu systému pro řešení celé agendy. Navrhli jsme obecnou architekturu systému pro vyhledávání osob dle kompetencí. Dále jsme popsali implementační část této práce, jejíž cílem bylo demonstrovat, že návrh je možné implementovat a skutečně funguje. Implementovaný systém pro vyhledávání kompetencí na FEL ČVUT je důkazem funkčnosti a po dalších rozšířeních může sloužit účelu, ke kterému byl navržen.\par
% Způsob jakým na tuto práci navázat a rozvinout její výstupy (implementační i teoretické) jsme shrnuli v příloze \ref{future}.