\chapter{Reprezentace dat} \label{chap:data_representation}
V této kapitole odpovíme na otázku \textit{Jak budeme data ukládat?}. Rozebereme tedy možnosti reprezentace jednotlivých entit z konceptuálního modelu (obr. \ref{fig:conceptual_appendix}). V závěru této sekce bude jasné, která reprezentace je pro náš případ nejvhodnější a proč.\par
Některé entity z konceptuálního modelu je možné reprezentovat poměrně jednoduše~-~osoby a pracoviště. Stačí nám na to objektové paradigma \cite{Wegner:object_oriented}, které se obvykle při modelování systémů používá.\par
Naproti tomu pracujeme s kompetencemi, znalostmi, dovednostmi a osobností. Modularita těchto entit je větší než u osob a pracovišť. Reprezentovat je klasickým objektovým způsobem může být velmi komplikované. U těchto entit je třeba jít do většího detailu. Reprezentaci osob a pracovišť můžeme později přizpůsobit dle zvolené reprezentace ostatních entit.\par
% \paragraph{Dílčí otázky}:
% \begin{itemize}
%     \item [] \textit{U kterých entit rozebereme reprezentaci dopodrobna?}
%     \item [] \textit{Jaké jsou varianty reprezentace těchto entit a jaké jsou jejich klady a zápory?}
%     \item [] \textit{Jaké existují praktické realizace - implementované formální jazyky, notace apod.?}
% \end{itemize}

\section{Kompetence}
Kompetence osoby, jak je vidět z konceptuálního modelu (příloha \ref{fig:conceptual_appendix}), podmiňují dva vlivy, osobnost a znalosti (resp. dovednosti). Tyto dvě oblasti určují jaké kompetence daná osoba má a v jaké míře.\par
\subsection{Vymezení}
Reprezentace osobnosti tvoří velmi komplikovanou problematiku, na toto téma odkážeme v sekci budoucí práce (kapitola \ref{future}), jelikož stojí za další zkoumání. V této práci se tímto však nebudeme více zabývat, místo toho se zaměříme na znalosti.\par
Omezíme se na reprezentaci kompetencí pomocí znalostí a dovedností a připodobníme míru kompetence k síle znalosti popř. dovednosti, z diagramu (příloha \ref{fig:conceptual_appendix}). 
% Přestože se jimi nebudeme přímo zabývat, budeme v této práci stále brát v úvahu pozdější rozšiřitelnost o další vlivy na kompetence.
\par
Dále problematiku zúžíme pouze na znalosti, jelikož z definice dovednosti plyne, že je závislá na znalosti. Není správné dovednosti úplně vyloučit, i na ně odkážeme v kapitole budoucí práce (kapitola \ref{future}). Společně se zkušenostmi by měly být dalším důležitým okruhem, který staví na znalostech a mimo jiné popisuje, jak je osoba umí využívat.\par
Otázku jsme tedy zúžili z reprezentace kompetencí na reprezentaci znalostí, na tu se zaměříme v další sekci.
\section{Znalosti} \label{sec:knowledge_representation}
Reprezentace znalostí (angl. \textit{knowledge representation}) je společně s uvožováním (angl. \textit{reasoning}) prudce se rozvíjejícím oborem umělé inteligence. V této sekci popíšeme, co vůbec reprezentace znalostí je a představíme si vybrané formální jazyky, které pro tuto reprezentaci lze využít (v některých zdrojích se mluví o formalismech, v jiných o formách, my jim budeme říkat formální jazyky, protože, to je zdaleka nejjednoznačnější pojmenování).
\subsubsection{Co je reprezentace znalostí}
Dříve než přistoupíme ke konkrétním formálním jazykům, musíme si ujasnit termín reprezentace znalosti. Reprezentace znalostí se dá shrnout do čtyř základních charakteristik:
\begin{itemize}
\item Náhrada reality
\item Množina ontologických závazků
\item Fragmentární teorie inteligentního uvažování
\item Prostředek pro počítač i člověka \cite{cite:09}
\end{itemize}
\subsubsection{Náhrada reality}
Pod touto charakteristikou leží základní filozofický předpoklad, že realita existuje nezávisle na jakémkoliv pozorovateli - počítači, člověku aj.\par
Náhrada reality, tedy reprezentace znalosti, nám umožňuje uvažovat a modelovat situace. Základem uvažování je vyvozování závěrů bez samotných činů. Příkladem může být plán stavby domu. Aniž bychom do ruky vzali jediný nástroj či materiál, jsme schopni dům zkonstruovat v rozsahu svých znalostí pouze ve vlastní hlavě.\par
Kvalita a rozsah naší reprezentace se vždy odvíjí od přesného účelu, pro který znalosti používáme/uchováváme. Je velmi nepravděpodobné, že by se nám podařilo reprezentovat naprosto všechno bez rozdílu od reality, je proto důležité vědět, jak moc a čím se reprezentace od reality liší.\cite{cite:10}
\subsubsection{Ontologické závazky} \label{ontological_commitments}
Předpokládáme, že nelze reprezentovat vždy všechno, každá reprezentace je v podstatě jen aproximací reality. Reprezentaci volí sám pozorovatel, tím dělá rozhodnutí, které znalosti bude reprezentovat a jak. Volba nejpříhodnější reprezentace je tedy množinou takových rozhodnutí - přesněji ontologických závazků.\cite{cite:09}
\subsubsection{Fragmentární teorie inteligentního uvažování}
Reprezentace znalostí je známka inteligentního chování. Existují nejméně dva pohledy na to, co je inteligentní chování. První z nich odkazuje na logiku a říká, že cokoliv, co se chová podle logických pravidel, je inteligentní. Druhý pohled nahrazuje logická pravidla množinou mechanizmů inspirovaných procesem poznáváním.\cite{cite:10} \par 
Oběma těmito přístupy se existující formální jazyky inspirují, některé využívají logiku, některé proces poznávání. \cite{cite:10}
\subsubsection{Prostředek pro počítač i člověka}
Počítačové zpracování znalostí reprezentovaných formálními jazyky pro to určenými musí být efektivní. Často platí, že čím je jazyk složitější, tím je složitější jej zpracovat.\cite{cite:10} \par 
Naproti tomu musí být jazyk dobře pochopitelný pro člověka. Zde se nabízí přirozený jazyk, ten však díky svým vlastnostem (kapitola č. \ref{sec:terms} sekce č. \ref{sec:jazyk}) není vhodný. Bude-li jazyk špatně pochopitelný pro člověka, nebude možné rychle ověřit pravdivost/důvěryhodnost sdělení vyjádřeného pomocí tohoto jazyka. Takový jazyk vědecká ani komerční komunita nepřijme. \cite{cite:10}

\section{Formální jazyky pro reprezentaci znalostí}
V této sekci shrneme existující formální jazyky pro reprezentaci znalostí. Každý z nich zevrubně popíšeme, abychom v závěru této kapitoly mohli rozhodnout, který se pro naše účely hodí nejvíce.

% \subsection{Rozdělení formálních jazyků}
% Formální jazyky pro reprezentaci znalostí lze rozdělit na dvě skupiny - \textit{logic-based} a \textit{non-logic-based}. (nevhodné překládat z angličtiny)\par
% \textit{Non-logic-based} jazyky nemají tak silné sémantické hranice, dá se říci, že jsou "méně formální". Jedná se například o sémantické sítě (popsány níže).\par
% % Naproti tomu \textit{Logic-based} mají díky logice sémantiku přesně danou. Můžeme tedy říct, že z hlediska vlastností formálních jazyků, popsaných v sekci \ref{sec:jazyk}, jsou "více formální". Jedná se například o jazyk ALC - jazyk deskripční logiky. (popsán níže) \par
% Speciálním případem jsou ontologie, které rozebíráme v dalších sekcí. Ontologie, jelikož staví na základech více jazyků pro reprezentaci znalostí, mají úroveň "formálnosti" modulárnější. Tato "formálnost" se často odvíjí od konkrétního případu užití a zvoleného způsobu řešení. 
\subsection{Sémantické sítě}
Sémantická síť je graf, jehož uzly reprezentují koncepty a individuality, hrany potom reprezentují vztahy mezi nimi.\cite{cite:10}\par
\begin{figure}[htbp!]
	\includegraphics[width=0.8\linewidth]{img/semantic_network.png}
	\caption{Příklad sémantické sítě popisující některé rodinné vztahy (zdroj \cite{Stephan2007}, přeložil autor)}
	\label{fig:semantic_network}
\end{figure}
\noindent \textbf{Individuality} reprezentují konkrétní objekty ze zkoumané domény, na obrázku č. \ref{fig:semantic_network} jsou to např. otec Stanislav nebo dítě Antonín.\par
\noindent \textbf{Koncepty} jsou množiny individualit. Na obrázku č. \ref{fig:semantic_network} jsou to např. rodič a osoba, které reprezentují všechny rodiče a všechny lidi ve sledované doméně.\cite{cite:10}\par
\par Každý uzel má své vlastnosti, jsou jimi atributy a relace. Atributem může být např. věk u osoby. Základními relacemi jsou IS-A a \textit{instance-of}. Relace IS-A definuje hierarchii konceptů (tvz. taxonomii), v obrázku č.\ref{fig:semantic_network} platí, že matka IS-A rodič. Vztah IS-A indikuje dědění všech vlastností nadřazeného konceptu konceptem podřízeným (např. matka dědí všechny vlastnosti rodiče). Vztah \textit{instance-of} je vždy mezi konceptem a individualitou, v obr. č. \ref{fig:semantic_network} platí Stanislav \textit{instance-of} otec.\cite{cite:10} Nejedná se samozřejmě o všechny druhy vazeb, jen o ty klíčové.\par
Sémantické sítě se nejčastěji používají pro modelování deklarativních znalostí, lze je však použít i pro procedurální znalosti.\cite{cite:11}\par
Původní význam slova \textit{sémantická sít} je \textit{fyzický model lidské paměti} \cite{cite:12}. Sémantické sítě se člověku snadněji chápou a práce s nimi je přirozenější než u dalších modelů, hlavně u těch logických.\par
Nevýhodou sémantických sítí je, že neexistuje jednotný standard pro interpretaci a jednotný standard pro hodnoty uzlů a hran - tedy neexistuje jednotná sémantika. Další nevýhodou je dědičnost, která nepodporuje žádné výjimečné stavy (např. tučňák je pták, tedy měl by mít všechny vlastnosti ptáků, ale nelétá). \cite{cite:12}
\subsection{Rámce}
Síť rámců (angl. frames) lze považovat za jeden z typů sémantických sítí. Vedle deklarativních znalostí, které reprezentuje obecná sémantická síť, reprezentují rámce i procedurální znalosti.\par
Základním stavebním kamenem této sítě jsou rámce. Ty v sobě uchovávají proceduru a skupinu atributů, které popisují danou situaci. Rámce se inspirují v lidské paměti, která si uchovává situace - kombinace procedurálních a deklarativních znalostí. Tak může lidský mozek například na dvě, na první pohled rozdílné, situace reagovat stejným způsobem, protože nalezne podobnost.\cite{cite:11}\par
Rámce řeší některé nedostatky sémantických sítí jako je: řešení výjimečných stavů, kontrola vnitřní konzistence a zamezení konfliktům atributů při dědičnosti. Hlavní nevýhodou je složitost použití.\cite{cite:12}
\subsection{Pravidla}
Deklarativní i procedurální znalosti lze reprezentovat jako množinu pravidel. Typický tvar těchto pravidel je tzv. JESTLIŽE-PAK (angl. IF-THEN), například \textit{jestliže je nalezena odpověď, pak jí přestaň hledat, jinak pokračuj v hledání}\cite{cite:11}. Takovou reprezentaci využívají např. logické programovací jazyky.\cite{cite:12}\par
Výhodou pravidel je, že jsou modulární a zároveň jsou poměrně prostorově nenáročná. Jednotlivá pravidla mohou být přidávána a odebírána bez ovlivnění zbylých. Mezi nevýhody patří hlavně složitost odhalení rozporu mezi pravidly a složitost reprezentace složitějších strukturovaných znalostí (svazuje nás přesná forma). V neposlední řadě je velmi složité předpovědět akce, které na základě pravidel nastanou, a tak se pravidla mohou nekonečně řetězit.\cite{cite:12}\par
Pravidla se samozřejmě (např. v programovacích jazycích) aktivně využívají, ale nehodí se pro reprezentaci komplexních znalostí. V programovacích jazycích se používají velmi hojně pro řízení toku programu, ale oproti reprezentaci celé zkoumané domény znalostí je to stále poměrně málo.
\subsection{Logika}
Logika mezi ostatními formálními jazyky pro reprezentaci znalostí převládá \cite{Stephan2007}. V této sekci projdeme klíčové koncepty logiky, nejdříve se zaměříme na obecnou rovinu, poté popíšeme více do detailu deskripční logiku.\par
Logika má mnoho forem, některé z nich se používají, nebo alespoň byly použity, pro reprezentaci znalostí. \cite{cite:12} Jen některé z těchto vyzkoušených forem se osvědčily. \par
V předchozích způsobech reprezentace, zejména potom u sémantických sítí, chyběla jednotně definovaná sémantika, proto je velmi komplikované tyto způsoby přímo aplikovat. \cite{Stephan2007} Znalosti vyjádřené pomocí sémantických sítí mohou být jednoduše vyjádřeny formou logiky, která sítím přidává potřebnou sémantiku. \cite{cite:12}\par
\noindent Následující dvě zobrazení reprezentují totéž:\par
\vspace{3mm}
\noindent\fbox{Zaměstnanec} $\xrightarrow{\text{\textit{je druhem}}}$ \fbox{Osoba} \indent
$\forall x: (Zaměstnanec(x) \to Člověk(x))$
\par\vspace{4mm}
\noindent Stejně jako je tomu u sémantických sítí, tak i u logiky platí, že jsme pomocí ní schopni vyjádřit naprostou většinu přirozeného jazyka. Je důležité si uvědomit, že přestože má logika v sémantice navrch nad sítěmi, sítě jsou intuitivnější pro lidské chápání. \par
Formální jazyky založené na logice obvykle zakládají na logice prvního řádu (predikátové logice), zkratka FOL.\cite{cite:12}\par

\subsubsection{Deskripční logika}
Deskripční logika je natolik expresivním jazykem, že se v dnešní době stala nejčastějším logickým paradigmatem pro reprezentaci znalostí, zejména potom v sémantickém webu. \cite{Stephan2007}.\par
Deskripční logika je téměř výhradně podmnožinou logiky prvního řádu (FOL), tedy logiky predikátové \cite{Stephan2007}. FOL popisuje doménu pomocí objektů (entity mající vlastní identitu), okolo těchto entit jsou konstruovány, pomocí funkcí, proměnných a logických spojek, logické vzorce (predikáty) \cite{Russell:2009:AIM:1671238}. Deskripční logika je typicky omezena pouze na unární a binární predikáty z FOL. \cite{Stephan2007}\par
\paragraph{Základními konstrukty jsou:}
\begin{itemize}
\item Atomické koncepty
\begin{itemize}
\item reprezentují pojmenované unární predikáty
\item např. \textit{Rodič}
\end{itemize}
\item Atomické role
\begin{itemize}
\item reprezentují pojmenované binární predikáty
\item např. \textit{má dítě, je dítětem}
\end{itemize}
\item Individuality
\begin{itemize}
\item reprezentují prvky konceptů, tzn. jednotlivce
\item např. \textit{Jiří} \cite{Kremen2011}
\end{itemize}
\end{itemize}
\paragraph{Znalostní báze většiny deskripčních logik se skládá z \cite{Stephan2007}:}
\begin{itemize}
\item \textit{T-Box} - reprezentuje axiomy obecně platné v dané doméně
\begin{itemize}
\item např. \textit{Muž je osoba}
\end{itemize}
\item \textit{A-Box} - reprezentuje konkrétní relační strukturu
\begin{itemize}
\item např. \textit{Jiří je muž} \cite{Kremen2011}
\end{itemize}
\end{itemize}
Jednotlivé deskripční logiky se liší v možnostech tvořit složitější koncepty případně role a v typech axiomů \cite{Kremen2011}.
Jeden z nejdůležitějších a nejpoužívanějších jazyků deskripční logiky je ALC. \cite{Stephan2007}

\section{Ontologie} \label{ontologies}
Všechny jazyky pro reprezentaci znalostí, které nejsou založeny na logice mají své nedostatky. Nejčastěji je problém v nedostatku sémantiky. Logika těmto jazykům potřebnou sémantiku dodala a můžeme jí díky tomu prohlásit za nejpřesnější způsob, jak znalosti formálně reprezentovat. (ne nutně nejefektivnější)\par
V této sekci přejdeme od konkrétních jazyků ke komplexnějšímu pohledu na reprezentaci znalostí. Ontologie lze také považovat za formální jazyk reprezentace znalostí. Jsou  však spíše realizací předchozích formálních jazyků než-li nový teoretický pohled na reprezentaci znalostí.\par
\subsection{Význam a Definice}
Význam slova ve filozofii odkazuje na tzv. "jsoucno", tedy zkoumání existence. Se zkoumáním existence úzce souvisí otázka: \textit{Co existuje?}. K odpovědi na tuto otázku je nutný systematický postup. Potřebujeme definovat tzv. ontologické kategorie, do kterých všechny zkoumané objekty rozdělíme. Příkladem takových ontologických kategorií mohou být: \textit{lidé, zvířata, věci}. Konkrétní existující instance 
% (nazveme je \textit{individuality} jako v případě předchozích formálních jazyků) 
jsou potom např. \textit{člověk Jiří, pes Alík, česká královská koruna}. Přesně z této ideje ontologických kategorií vychází význam slova ontologie v počítačových vědách. \cite{Stephan2007} Nejedná se tedy o stejné zkoumání jako v případě filosofie, ale stále se pohybujeme ve stejné oblasti.\par
Ontologickými kategoriemi obecně určujeme, které znalosti budeme schopni reprezentovat - určujeme specifickou doménu, pro kterou znalosti modelujeme. Tvoříme tak tzv. \textit{ontologický slovník}.\par
% Zde je možné najít souvislost s ostatními formálními jazyky - ontologie v podstatě popisují názvy hran v sémantických sítích, popř. logické vzorce, používané pro reprezentaci znalostí.\cite{Stephan2007}\par
Obecně ontologie tedy popisují, co v dané doméně existuje. Ontologie jako konkrétní počítačový artefakt (např. soubor) kóduje danou znalost o doméně v počítačem zpracovatelné formě.\cite{Stephan2007}\par
\noindent Doslovná definice ontologie dle \cite{gruber1993translation} je:\par
\begin{quote}
    \textit{Ontologie je explicitní specifikace konceptualizace.}
\end{quote}\par
\noindent Později byla tato definice rozšířena dle \cite{Borst97} na:\par
\begin{quote}
    \textit{Ontologie je formální specifikace sdílené konceptualizace.}
\end{quote}\par
\subsection{Základní charakteristiky}
\begin{enumerate}
\item Formální
\begin{itemize}
\item Ontologie musí být vyjádřena formálním jazykem (výhody těchto jazyků jsou popsány v kapitole č. \ref{sec:terms} sekce č. \ref{sec:jazyk}).
\end{itemize}
\item Explicitní
\begin{itemize}
\item Znalost musí být vyjádřena maximálně explicitně, pro optimální strojové zpracování.
\end{itemize}
\item Sdílené
\begin{itemize}
\item Ontologie vždy reflektuje dohodu skupiny lidí v dané doméně, v podstatě se jedná o kolektivní ontologické závazky (viz sekce č. \ref{ontological_commitments}).
\end{itemize}
\item Konceptuální
\begin{itemize}
\item Ontologie popisuje znalosti pomocí symbolů, které reprezentují koncepty a relace mezi nimi. Konceptualita spočívá v tom, že jsou ontologie intuitivně uchopitelné pro člověka, protože odpovídají lidskému mentálnímu modelu znalostí. Jako protipříklad lze uvést neuronové sítě, které též reprezentují určité znalosti, ale přirozenému lidskému chápání je to již vzdáleno.
\end{itemize}
\item Doménově specifická
\begin{itemize}
\item Ontologie je vždy omezena na určitou oblast - doménu.\cite{Stephan2007}
\end{itemize}
\end{enumerate}

% Ontologie jsou v praxi používány v informačních systémech pro rozhodování a vyvozování (angl. reasoning) v dané doméně.\cite{Stephan2007}\par

\subsection{Základní stavební kameny}
\begin{itemize}
\item Koncepty
\begin{itemize}
\item V sémantických sítích i v deksripční logice se tento konstrukt nazývá stejně
\item Reprezentují ontologické kategorie
% \item V příloze na obrázku č. \ref{fig:ontology_example} např. City
\end{itemize}
\item Relace
\begin{itemize}
\item Analogicky se jedná o hrany sémantické sítě nebo role v deskripční logice
\item Propojují koncepty a instance - reprezentují vazby mezi nimi
% \item V příloze na č. \ref{fig:ontology_example} např. locatedIn
\end{itemize}
\item Instance
\begin{itemize}
\item Individuality v sémantické síti nebo v deskripční logice
\item Reprezentují konkrétní objekty v doméně \cite{Stephan2007}
% \item V příloze na obrázku č. \ref{fig:ontology_example} např. Berlin
\end{itemize}
\end{itemize}
%Ontologii si lze také představit jako množinu výroků (axiomů) např. \textit{OsobaX je člověk} apod.  \cite{Stephan2007}\par
\subsection{Zápis ontologií}
Ontologie lze zapisovat několika způsoby. Například pomocí sémantických sítí, které typicky neobsahují všechny důležité informace, jelikož nejsou tak expresivní (viz sekce č. \ref{fig:semantic_network}). Toto zobrazení je příznivé hlavně pro pochopení lidmi. Dále je zapisujeme pomocí serializovaného formátu (např. XML), vhodného pro strojové zpracování. Dále pomocí logických vzorců, to je vhodné pro další vyvozování (angl. reasoning). \cite{Stephan2007}.

\subsection{Použití ontologií}
\begin{itemize}
\item \textbf{Propojení znalostí} - jednotný znalostní slovník napříč aplikacemi, datový zdroj znalostí
\item \textbf{Abstrakce znalostí} - zdroj schématu znalostí (hierarchie a vztahy konceptů) pro budoucí využití
\item \textbf{Automatizace zpracování znalostí} - automatické vyvozování závěrů z platných axiomů (angl. reasoning)
\item \textbf{Integrace informací} - integrace na úrovni různých datových schémat a interpretace získaných dat z jednoho zdroje v rozdílném schématu
\item \textbf{Získávání informací} - zdokonalování modelu znalostí s důrazem na jeho prohledávání
\item \textbf{Sémantická správa obsahu} - přidávání metadat k datům, která později slouží k jejich lepší identifikaci a lepšímu přístupu k nim
\item \textbf{Organizace znalostí} - sdílený znalostní model daného seskupení lidí (např. firma, škola)
\item \textbf{Expertní systémy} - systémy schopné řešit komplikované otázky (úlohy) - např. i v medicíně \cite{Stephan2007}
\end{itemize}
\subsection{Typy ontologií}
Existují čtyři základní typy ontologií. Jednotlivé typy se od sebe liší úrovní abstrakce reprezentovaných znalostí. \cite{Stephan2007}\par
\begin{itemize}
\item \textbf{Top-level}
\begin{itemize}
\item Abstraktní a obecné koncepty (např. \textit{fyzický objekt, abstraktní objekt}) s velkým potenciálem znovupoužití v mnoha doménách
% \item typicky se přímo nenachází v aplikacích - spíše ontologie, které se jimi řídí
\item Typicky se používají jako předloha pro tvorbu ontologií zmíněných níže
\item Příklady UFO \cite{Guizzardi2005}, SUMO \cite{Niles:2001:TSU:505168.505170}, DOLCE \cite{Dolce}
\end{itemize}
\item \textbf{Doménové a úkolové ontologie}
\begin{itemize}
\item Zachytávají znalost ve specifické doméně - např. medicíně, geografii; nebo znalost o specifickém úkolu - např. diagnostika, konfigurace
% \item Doménové ontologie se zaměřují na "statické" znalosti o doméně, úkolové potom na postupy, plánování, monitoring apod. - "dynamické" znalosti
\end{itemize}
\item \textbf{Aplikační ontologie}
\begin{itemize}
\item Propojují doménové a úkolové ontologie pro účely konkrétní aplikace
\end{itemize}
\end{itemize}
Jedná se o inkluzivní hierarchii, nižší ontologie přejímají a konkretizují obecné koncepty a relace vyšších ontologií. \cite{Stephan2007}
\begin{figure}[htbp!]
	\includegraphics[width=0.7\linewidth]{img/ontology_types.png}
	\caption{Typy ontologií (zdroj \cite{Stephan2007}, přeložil autor)}
	\label{fig:ontology_types}
\end{figure}


\subsection{Přidružené technologie, standardy a další}
% Existuje několik jazyků, které jsou určené pro zápis a jednoznačnou reprezentaci ontologií ve strojově čitelné formě. Uvedeny jsou pouze ti nejpoužívanější kandidáti.
\subsubsection{RDF}
V doslovném překladu se jedná o framework pro popis zdrojů. RDF je standardizovaný jazyk pro reprezentaci znalostí (informací) na webu. Tento jazyk lze však použít pro reprezentaci libovolných znalostí.\par
\paragraph{Základními elementy jsou:}
\begin{itemize}
\item URI (Uniformní identifikátor zdroje) - slouží k pojmenování a identifikaci individualit, konceptů a jejich vlastností (příklad je \textit{urn:isbn:0451450523}, což je URI knihy)
\item Věty ve tvaru: \fbox{Podmět} $\xrightarrow{\text{\textit{Predikát}}}$ \fbox{Předmět} (jednotlivé členy jsou ve tvaru URI)
% \begin{itemize}
% \item []\textbf{Příklad} \fbox{http://web.de/MrX.html}$\xrightarrow{\text{\textit{btr:maAutora}}}$ \fbox{btr:PanX} \cite{Stephan2007}
% \end{itemize}
\end{itemize}
RDF je typicky zpracováváno v serializované podobě v XML formátu (viz příklad níže) \cite{Stephan2007}. Poměrně populární variantou pro serializaci RDF je též JSON-LD \cite{JSON_LD} nebo Turtle \cite{TURTLE}.\par

\begin{lstlisting}[caption= Příklad syntaxe RDF XML, captionpos=b]
<rdf:Description rdf:about="http://ubiqbiz.com/web/MrX.html">
    <btr:hasAuthor rdf:resource="btr:PanX"/>
</rdf:Description>
<rdf:Description rdf:about="btr:PanX">
    <btr:employedAt rdf:resource="btr:UbiqBiz"/>
</rdf:Description>
\end{lstlisting} 

RDF je framework pro prostý popis zdrojů, pokud chceme datům dávat význam a s tímto významem pracovat, musíme využít další rozšíření. Typicky datům dodáváme soubor axiomů pro další vyvozování (angl. inference) popř. uvažování (angl. reasoning), které nám umožní je dále rozvíjet a efektivně s nimi pracovat. V následujícím seznamu uvádíme příklady takových rozšíření.
\begin{itemize}
\item Vyvozování dalších znalostí, popis schématu
\begin{itemize}
\item \textbf{RDFS} - omezené prostředky na rozdíl od pokročilejších \par(zdroj: \url{https://www.w3.org/TR/rdf-schema/})
\item \textbf{OWL} - ve své podstatě univerzální jazyk pro popis ontologií \par(zdroj: \url{https://www.w3.org/TR/owl2-primer/}, podrobněji níže)
\end{itemize}
\item Validace modelu
\begin{itemize}
\item \textbf{SHACL} - validace RDFS (zdroj: \url{https://www.w3.org/TR/shacl/})
\end{itemize}
\end{itemize}

\subsubsection{SPARQL}
\noindent Dotazovací jazyk standardizovaný pro vyhledávání a další operace v RDF \cite{SPARQL}.\par
\subsubsection{OWL}
\noindent Web Ontology Language (OWL) je  standardizovaným jazykem pro sémantickou anotaci webového obsahu. Jazyk má 3 varianty, my se zaměříme pouze na nejčastěji používanou a to je OWL-DL (z angl. \textit{OWL description logic}).\par
Syntaxe OWL zakládá na RDF a rozšiřuje jej. Větám (trojicím) dává ve své podstatě další sémantiku, kterou lze později interpretovat.\par
% \paragraph{Základními konstrukty jsou:}
% \begin{itemize}
% \item Třídy (koncepty z DL)
% \item Vlastnosti (role z DL)
% \item Individuality (v DL stejně)
% \item Konstruktory tříd (slouží k tvorbě složitějších tříd)
% \end{itemize}
Dokument v jazyce OWL je v podstatě množina výroků, které mohou být interpretovány jako axiomy v deskripční logice. OWL je mocný nástroj, jelikož vychází z deskripční logiky a převzalo její výhody - je možné kontrolovat konzistenci a vyvozovat implicitní znalosti. \cite{Stephan2007}

\subsubsection{OntoUML}
Rozšíření jazyka UML určené pro modelování ontologií. Toto rozšíření je založeno na top-level ontologii UFO \cite{Guizzardi2005}.

\subsubsection{SKOS}
SKOS (z angl. Simple knowledge organization system) je standard pro sdílení strojově čitelných dat. Tento standard řeší základní problém systémů pro organizaci znalostí (zkratka KOS). Tyto systémy operují se znalostmi v různé formě~-~např. obecné slovníky, thesauri, taxonomie, klasifikační schémata. SKOS standardizoval sdílení těchto dat v jednotné formě, která zachovává výhody původních forem (slovníky, thesauri...). Cílem je mít data dostupná v jednotném formátu - RDF. \cite{SKOS}\par 
% OWL Full ontologie \cite{OWL_FULL}.
Standard SKOS klade menší důraz na formálnost, míří více na přímou konverzi dat na znalosti a jejich použití v různých situacích. Neklade takový důraz na uvažování (angl. reasoning) a vysokou přesnost dat. \cite{isaac2009skos}\par 
SKOS je ve své podstatě ontologií, jak je možné vidět na obrázku v příloze \ref{fig:SKOS-ontology}. 

% Bližší specifikace SKOS je dostupná zde https://www.w3.org/TR/2005/WD-swbp-skos-core-guide-20051102/#secabout.\cite{SKOS}

%TODO příklad OWL














\subsection{Reprezentace a datová struktura}
Každá reprezentace je nakonec realizována pomocí některé známé datové struktury. Například u sémantické sítě je to typicky graf. Je důležité striktně dodržovat sémantiku dané reprezentace, která ne vždy platí obecně pro použitou datovou strukturu. Například sémantická síť popisující rodinné vztahy v sobě netoleruje cykly, to ale pro obecný graf neplatí. \cite{cite:09}




% TODO: 
% jednotlivé zhodnotit - napsat vhodné použití apod., zkusit se zaměřit na ohledy vyhledávání v těchto informacích pomocí fulltextu...(snad)



\section{Shrnutí}
Cílem této kapitoly bylo odpovědět na další z dílčích otázek našeho problému: \textit{Jak budeme data ukládat?}. Rozebrali jsme tedy možnosti reprezentace jednotlivých entit a nyní můžeme rozhodnout, která je pro naše účely nejvhodnější.\par 
Je důležité zmínit, že námi zvolený postup jistě není jediný možný. Přestože v této kapitole vybereme jednu z představených možností, nadále budeme pracovat tak, aby naše řešení bylo modulární. Budeme jej navrhovat tak, aby jeho části bylo možné nahrazovat vylepšenými případně změněnými (hlavně při návrhu systému - kapitola č. \ref{chap:app-design}).\par
Klíčovou částí této kapitoly byly formální jazyky pro reprezentaci znalostí. Jazyky byly představeny s vzestupnou tendencí, co se týká praktického použití. První představené, hlavně sémantické sítě, byly lépe pochopitelné pro člověka, ale postrádaly dostatečně definovanou sémantiku pro praktické aplikace. Naproti tomu později představená deskripční logika je kvalitnějším jazykem, jelikož předchozím formálním jazykům dodává zmíněnou potřebnou sémantiku.\par
Nakonec jsme se v této kapitole zabývali ontologiemi. Ontologie jsou více realizací zmíněných formálních jazyků, než že by do této problematiky přinášeli něco úplně nového. Ontologie staví na deskripční logice a přebírají expresivitu tohoto jazyka, zároveň jsou vizualizovány pomocí sémantických sítí, které jsou pro práci přehlednější pro člověka. V kontextu ontologií existují konkrétní jazyky (RDF, OWL) a praktické aplikace (SKOS), které jsou již ověřené.\par
Vzhledem k vlastnostem, které jsme zmínili, jsou ontologie a množina přilehlých technologií v našem případě vhodnou volbou pro reprezentaci znalostí.\par

% Volba ontologie, pro tuto práci, shrnuli jsme však teoretický základ pro další směry
% Zmínit se o osobách a pracovištích

% Následujícím úkolem je zajistit, jak data, která máme k dispozici, transformovat do zvolených reprezentací - ontologií. Této výzvě však předchází ještě analýza datových zdrojů, která by nám měla ukázat, jakou povahu naše data budou typicky mít. Na základě této znalosti budeme schopni rozhodnout o transformaci. Na datové zdroje se zaměříme v další kapitole.

% POZOR - měli bychom si uvědomit nejdříve proč ty data chceme vůbec reprezentovat - to nám v pozdější sekci pomůže říct, co je pro nás lepší (https://www.matthewrenze.com/presentations/transforming-data-into-knowledge.pdf) Náš účel bude data exploration (úzce propojeno s vizualizací.
% \section{Osoby a pracoviště}
% Tyto dvě třídy mají pouze prosté atributy a neměl by být problém je reprezentovat různými způsoby. Osoba samozřejmě vlastní kompetence, znalosti a osobnost, proto bude n

