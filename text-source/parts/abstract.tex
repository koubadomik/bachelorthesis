% Abstrakt v anglictine
\begin{abstract-english}
         In this work, we dealt with competency-based people and workplace search at the CTU FEE.\par
First, we discussed the problem in general. In this part of the thesis, we decided that ontology would be a suitable representation of competencies. Mainly because, as the only formal language for knowledge representation, it has practical support for technologies such as RDF, OWL and SKOS. Furthermore, we analyzed possible data sources, especially the sources of people knowledge and the knowledge base. Finally, we dealt with data processing. We analyzed transformation into ontologies using tools such as OntoRefine. Then we described work with SPARQL.\par
In the second part of the thesis, we have designed a system that solves the competency-based search. We decided to use the hexagonal architecture, which is sufficiently modular for this use case. Moreover, in the second part of the thesis, we dealt with the situation at the CTU FEE. First, we discussed the data sources of knowledge, people and workplaces. We found Usermap and V3S the most important systems.\par
Last part of this work is a prototype, on which we demonstrate the suitable design of the system. We used existing \textit{triple-store} GraphDB with SKOS ontology extended by several entities to store ontological data. The server part of the prototype is implemented in JAVA using the Spring Boot framework. The prototype also includes a client application that demonstrates the communication with the server side. At the end of this work, we discussed the testing of hexagonal architecture and applied some of the presented techniques to the prototype.
\newpage
\end{abstract-english}


% Abstrakt v cestine
\begin{abstract-czech}
    	V této práci jsme se zabývali vyhledáváním osob a pracovišť dle kompetencí na FEL ČVUT.\par
    	Nejdříve jsme celý problém rozebrali obecně. V této části práce jsme rozhodli, že vhodnou reprezentací kompetencí budou ontologie. Hlavně proto, že jako jediný formální jazyk pro reprezentaci znalostí mají praktickou podporu technologií jako jsou RDF, OWL a SKOS. Dále jsme zanalyzovali možné datové zdroje, zejména zdroje znalostí osob a znalostní báze. Nakonec jsme se zabývali zpracováním dat, rozebrali jsme jejich transformaci do ontologií pomocí nástrojů jako je OntoRefine a následnou práci s nimi za pomoci jazyka SPARQL.\par
    	V druhé části práce jsme navrhli systém, který problematiku vyhledávání dle kompetencí řeší. Využili jsme k tomu hexagonální architekturu, která je pro náš případ užití dostatečně modulární. Dále jsme se v druhé části práce věnovali situaci na FEL ČVUT. Nejdříve jsme rozebrali datové zdroje znalostí, osob a pracovišť. Jako nejvýznamnější jsme vyhodnotili systémy Usermap a V3S.\par 
    	Součástí této práce je také implementovaný prototyp, na kterém demonstrujeme funkčnost navrženého systému. Pro ukládání dat pomocí ontologií jsme využili existující \textit{triple-store} GraphDB, jako schéma jsme použili existující ontologii SKOS rozšířenou o několik našich entit. Serverová část prototypu je implementovaná v jazyce JAVA za použití frameworku Spring Boot. Součástí prototypu je také klientská aplikace, která demonstruje průběh komunikace se serverem. V závěru této práce jsme rozebrali testování hexagonální architektury a některé z představených technik jsme aplikovali i na prototyp.
    	\newpage
    	
    % 	Celý problém jsme rozebrali nejdříve obecně, následně jsme navrhli systém, který tuto problematiku řeší.\par
    % 	Pro reprezentaci kompetencí jsme se rozhodli využít ontologie. Následně jsme zanalyzovali různé zdroje dat, které jsou pro náš případ potřeba
    % 	Na prototypu, který jsme v rámci této práce implementovali, demonstrujeme, že navržený systém lze po dalších rozšířeních použít v praxi. 
    % 	Navržený systém je vhodný například pro průmyslové partnery FEL ČVUT, kteří hledají odborníky v konkrétním oboru.
\end{abstract-czech}